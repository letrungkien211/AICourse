\documentclass[11pt]{article}
\title{Artificial Intelligence Report 1st Week-2nd Problem}
\author{Le Kien Trung \\
03-120291, 3rd Year Student, Department of Mechano-Informatics }
\date{November 1st, 2012}
\begin{document}
\maketitle

Human beings are distinguished from other living subjects or machines because of our advanced ability to perceive and interact with the environment. The complexity of our brain's structures is still too hard for modern science to fathom. That difference, I believe, is the main reason why machine would never achieve the same level with human brain. Even though, machines defeat human in numerous tasks such as calculation, chess, in term of both speed and accuracy, machines cannot and will never be able to match human ability to do thinking, reasoning and perceiving the world. \\

Let us first discuss about how human and machine are different from each other in term of hardware. Because our concern here is intelligence, obviously human parts other than brain will not be considered. The brain is the center of the nervous system that controls both our emotional and physical actions. It consists of billions of billions neuron, which are interconnected elegantly in order to tackle from very simple tasks such as doing a quick addition of 1 and 1 to extremely difficult ones such as proving General Euler's theory [1]. Considering to build a machine that could do that is definitely deemed ridiculous. Machine is built from electronic devices, whose structures seem to be very complex to us, but still not anything compare to human or even other living animals' brains. Many people can argue that human evolution process took a couple of million years, so computer probably will catch up with us after the same amount of time. It certainly is a good point, but I am not convinced. The bottom line is, while human brains are biologically constructed, computers are electronically built. Therefore, I believe computers will never achieve the same level of intelligence human beings possess. For example, living creates in general, and human in particular have the ability to pass their traits to their successors. Two great scientists can possibly have a child that change the world. Computer hardware has no such things, we cannot take two computers as inputs and output a new machine that carries some characteristics from its mothers. In fact, that natural evolution has been mimicked by the well-known genetic algorithm. However, that is not about hardware, but software or algorithm, which we will discuss about through more sophisticated example.\\

In Machine Learning and Artificial Intelligence, neural network has emerged as a state-of-the-art method which is not only beautiful from the idea of mimicking the human brain but also very powerful indeed. I have myself learnt a lot about artificial neural network from theory to practical implementation. Its intention is to solve artificial intelligence problems without actual biological brain. Just like human brains, it consists of artificial neurons that are connected or functionally related in a nervous system. It can be trained to recognize patterns such as handwriting, animals or predict future values such as next month housing price. However, it merely utilizes a tiny portion of human brains' functional structures. More importantly, artificial neural network has not been constructed in hardware level. Imagine, if we can build a computer chip which consists of neurons functioning just like the software's version, we can certainly do expect a lot more about reconstructing human brains. However, I doubt that will ever be built. How a hardware can contain a billion parts, each of which represent a neuron. How can they be connected in a way that would ensure they function properly. That is simply impossible. Still, hypothetically, we can ignore hardware parts, focus in software and algorithm only and hope that will help us to build a machine that surpass human thinking ability. However, computer's speed depends on hardware rather than software. In order to make a computer to think like a human, high speed is indispensable. A neural network built from hardware's level surely has much more potential than the software's one.  \\

Now I want to set aside the speed problem and consider the debatable Turing test. Supposed that we can build a super computer which can compute anything in reasonable time. That means it has an infinite memory store and speed. Up until now, I have stated or implied many times that computer would never reach the same intelligence's level of human beings, but I did not talk about what intelligence exactly is. How do we define an intelligent computer in contrast to normal ones. This problem was forcefully argued in a famous article by Alan Turing, entitled "Computing Machinery and Intelligence", which appeared in the philosophical journal Mind in 1950. In the article, the Turing test was described as a way to judge if a machine can reasonably be said to think. According to the Turing test, the computer and a human volunteer are both to be hidden from the view of some interrogator. The interrogator then has to decide which is the computer, which is human volunteer only by asking questions and receiving answers from both sides. The way questions and answers are presented must be impersonal, such as typing and screen's displaying. The computer is deemed to think if the interrogator cannot distinguish it from the human volunteer after a series of questions. So, a computer would be able to think if it pass the test. Is that sound too simple? In this case, the ability of thinking is confined in ability to answer questions. Imagine, you are an interviewer for a big company and you conducted interview with two people. You asked them the same question, they both were able to answer, but later you found out that one merely remembered the answer without any understanding while the other solve the problem by himself. What would you think? I certainly will prefer the later, as he was able to "think", the other was merely had good memory. In my opinion, the Turing test is not qualified for assessing computer's thinking ability. I also want to mention the Chinese room experiment to consolidate my argument. Supposed that there is a program that gives a computer the ability to carry on an intelligent conversation in written Chinese, which must involves ideal hardware. The program is given to a person who speak only English. If an Chinese speak to that person via internet (yahoo, gmail), he would give an easy conclusion that the person he is chatting with knows Chinese very well. However, that certainly is not the case, as even though he can chat with an Chinese, he or the computer do not understand a word in that conversation. The experiment is intended to argue that it is impossible to build a thinkable machine, no matter how intelligent we can make it appear to be. I agree, in fact, modern computer can do much better than us in numerous tasks, but we do not consider them to think when they conduct those tasks. In order to possess the ability to actually think and perceive the world, computer would need the same level of structure of human brain, which would never be possible. \\

In conclusion, I am convinced that the difference of hardware ensure that computers will never reach or surpass human thinking ability. Human beings are able to understand our brains' structure, so how could we build a machine that has the same function with the mysterious brains.  \\

\textbf{Reference} \\
1. The Emperor's New Mind by Roger Penrose, Oxford University Press. \\
2. Wikipedia: Turing test, Chinese room, Neural network
\textbf{Note} \\
Please ignore the Report 2 I sent on October 29, it was totally a mistake.
\end{document}




























